\documentclass[a4paper,11pt]{article}
\usepackage[utf8]{inputenc}
\usepackage[spanish]{babel}
\usepackage{amsmath}
\usepackage{amssymb}
\usepackage{cancel}
\usepackage{siunitx}
%para evitar separacion silabica al final de linea
\tolerance=1
\emergencystretch=\maxdimen
    \usepackage{subfigure}
\hyphenpenalty=10000
\hbadness=10000
\usepackage{vmargin}
	\setpapersize{A4}
	\setmarginsrb
		{2cm} %margen izquierdo
		{1cm} %margen superior (distancia del borde superior al encabezado)
		{2cm} %margen derecho
		{1cm} %margen inferior (distancia del borde inferior al pie)
		{0cm} %altura del encabezado
		{2cm} %espacio entre el texto y el encabezado de pagina (para hacer los 3cm)
		{0cm} %altura del pie de pagina
		{2cm} %espacio entre el texto y el pie de pagina (para hacer los 3cm)
\usepackage{natbib}
\usepackage{graphicx}
\usepackage{wrapfig}
\usepackage{steinmetz}

\usepackage{titlesec}
\titleformat{\section}
{\normalfont\Large\bfseries}{Problema~\thesection}{1em}{}
\titleformat{\subsection}
{\normalfont\large\bfseries}{Solución:}{1em}{}

\renewcommand{\theenumi}{\alph{enumi}} 

\title{Problemas de corriente alterna trifásica}

\date{}

\author{ETSIDI}

%%%%% EMPIEZA EL DOCUMENTO

\begin{document}

\maketitle

%%%%% PROBLEMAS

\section{} 
\begin{wrapfigure}[6]{r}{0.35\textwidth}
		\centering
		\includegraphics[width=0.35\textwidth]{figuras/ej1_CAtrif.pdf}
	\end{wrapfigure}
En el sistema de la figura de secuencia de fases directa y frecuencia $f=\SI{60}{\hertz}$, se dispone de un receptor equilibrado con una potencia total $P_T=\SI{51984}{\watt}$ factor de potencia de $0.6$ en retraso. Sabiendo que el amperímetro marca $\SI[parse-numbers=false]{76\sqrt{3}}{\ampere}$, determinar:
\begin{enumerate}
    \item Medida delos vatímetros 1 y 2.
    \item Valor de la impedancia $\overline{Z}$ en forma módulo-argumento.
    \item Valor de la capacidad mínima para mejorar el factor de potencia a $0.95$ en retraso.
    \item Valor de la impedancia equivalente en estrella.
\end{enumerate}

\subsection{}

%% SOLUCION
Para solucionar las preguntas en este problema, antes de calcular nada, podemos extraer la siguiente información del enunciado:
\begin{itemize} % para incluir viñetas
    \item Se tiene una sola carga trifásica equilibrada de valor $Z$ y con $\cos{\phi}=0,6\rightarrow \SI{53.13}{\degree}$ en retraso. Esto significa que la impedancia $\overline{Z}$ es de carácter inductivo y su potencia reactiva será positiva.
    \item 	Se tiene una secuencia de fases directa ABC. Esto significa que el sistema de alimentación tiene las siguientes tensiones de línea: $\overline{U_{AB}}=\SI[parse-numbers=false]{U_{AB}\phase{\SI{120}{\degree}}}{\volt}$, $\overline{U_{BC}}=\SI[parse-numbers=false]{U_{BC}\phase{\SI{0}{\degree}}}{\volt}$ y $\overline{U_{CA}}=\SI[parse-numbers=false]{U_{CA}\phase{\SI{-120}{\degree}}}{\volt}$. Así pues, las tensiones de fase son: $\overline{U_{A}}=\SI[parse-numbers=false]{U_{A}\phase{\SI{90}{\degree}}}{\volt}$, $\overline{U_{B}}=\SI[parse-numbers=false]{U_{B}\phase{\SI{-30}{\degree}}}{\volt}$ y $\overline{U_{C}}=\SI[parse-numbers=false]{U_{C}\phase{\SI{-150}{\degree}}}{\volt}$.
    \item Anotamos la frecuencia de red de valor $\SI{60}{\hertz}$ por si hemos de calcular alguna reactancia inductiva y/o capacitiva.
    \item La potencia activa total que demanda el triángulo de impedancia $\overline{Z}$ es de valor $P_T=\SI{51984}{\watt}$. De este valor, sacamos como conclusión que cada impedancia $\overline{Z}$ del triángulo consume un tercio de dicha potencia activa al ser un receptor equilibrado.
    \item El vatímetro $W_2$ está conectado midiendo la intensidad $\overline{I_{BC}}$ y la tensión $\overline{U_{BC}}$, es decir, nos da el valor de la potencia activa que disipa la fase BC del triángulo, cuyo valor ya sabemos que es:
    \[
        W_2=\dfrac{P_T}{3}=\dfrac{51984}{3}=\SI{17328}{\watt}
    \]
    \item El amperímetro dispuesto en la línea A mide el valor eficaz de $\SI[parse-numbers=false]{76\sqrt{3}}{\ampere}$. Esto significa que, al tener un receptor equilibrado conectado en triángulo, las intensidades por las otras dos líneas B y C tienen el mismo valor de intensidad de valor eficaz de $\SI[parse-numbers=false]{76\sqrt{3}}{\ampere}$.
    \item También, al ser un receptor equilibrado conectado en triángulo, las intensidades $\overline{I_{AB}}$, $\overline{I_{BC}}$ e $\overline{I_{CA}}$ que circulan dentro del triángulo, toman por valor eficaz:
    \[
    \dfrac{76\sqrt{3}}{\sqrt{3}}=\SI{76}{\ampere}
    \]
    \item 	Los argumentos de las intensidades dentro de triángulo se pueden obtener del propio enunciado. Cada intensidad $\overline{I_{AB}}$, $\overline{I_{BC}}$ e $\overline{I_{CA}}$ retrasa $\SI{53,13}{\degree}$ a las tensiones $\overline{U_{AB}}$, $\overline{U_{BC}}$ y $\overline{U_{CA}}$ correspondientes, es decir, la intensidad $\overline{I_{AB}}$ tiene un argumento de valor $120-53,13=\SI{66,87}{\degree}$, la intensidad $\overline{I_{BC}}$ tiene un argumento de valor $0-53,13=\SI{-53,13}{\degree}$ y la intensidad $\overline{I_{CA}}$ tiene un argumento de valor $-120-53,13=\SI{-153,13}{\degree}$.
    \item 	Los argumentos de las intensidades de línea también se pueden obtener del propio enunciado. Cada intensidad $\overline{I_A}$, $\overline{I_B}$ e $\overline{I_C}$ retrasa $\SI{53,13}{\degree}$ a las tensiones del sistema de alimentación $\overline{U_A}$, $\overline{U_B}$ y $\overline{U_C}$ correspondientes, es decir, la intensidad $\overline{I_A}$ tiene un argumento de valor $90-53,13=\SI{36,87}{\degree}$, la intensidad $\overline{I_B}$ tiene un argumento de valor $-30-53,13=\SI{-83,13}{\degree}$ y la intensidad $\overline{I_C}$ tiene un argumento de valor $-150-53,13=\SI{-203,13}{\degree}$.
\end{itemize}

Tras estas consideraciones, se pueden iniciar los cálculos necesarios para responder a las preguntas del problema:
\begin{enumerate}
    \item 	Medida de los vatímetros 1 y 2.
    
La lectura del vatímetro 1, según está conectado, es la siguiente:
\[
[W_1]=\overline{U_{AC}}\cdot \overline{I _A}=U_{AC}\cdot I_A\cdot \cos(\langle \overline{U_{AC}}, \overline{I_A} \rangle)
\]

Se desconoce el módulo de la tensión $\overline{U_{AC}}$. Se calcula a partir del vatímetro 2 cuya lectura es de $[W_2]=\SI{17328}{\watt}$:
\[
[W_2]=\overline{U_{BC}}\cdot \overline{I_{BC}}=U_{BC}\cdot I_{BC}\cdot \cos(\langle \overline{U_{BC}}; \overline{I_{BC}}\rangle);\; 17328=U_{BC}\cdot 76\cdot 0.6\Rightarrow U_{BC}=\SI{380}{\volt}
\]
Por tanto, al ser un sistema equilibrado ($U_{AB}=U_{BC}=U_{CA}$), y sabiendo que $\overline{U_{AC}}=-\overline{U_{CA}}=U_{CA}\phase{-120+180}=\SI[parse-numbers=false]{380\phase{\SI{60}{\degree}}}{\volt}$, la lectura del vatímetro 1:
\[
[W_1]=\overline{U_{AC}}\cdot \overline{I_A}=U_{AC}\cdot I_A\cdot \cos(\langle \overline{U_{AC}}, \overline{I_A} \rangle)=380\cdot 76\sqrt{3}\cdot \cos(\langle60;36.87\rangle)=\SI{46001}{\watt}
\]

\item Valor de la impedancia $\overline{Z}$ en forma módulo-argumento.

Al conocer ya el valor de la tensión a la que está alimentada y la intensidad que circula por ella, se obtiene su valor fácilmente a partir de la ley de Ohm:
\[
\overline{Z}=\dfrac{\overline{U_{AB}}}{\overline{I_{AB}}}=\dfrac{380\phase{120}}{76\phase{66.87}}=\SI[parse-numbers=false]{5\phase{\SI{53.13}{\degree}}}{\ohm}
\]

\item Valor de la capacidad mínima para mejorar el factor de potencia a $0.95$ en retraso.




    \item Valor de la impedancia equivalente en estrella.


\end{enumerate}

\section{} 
% poner aqui el ENUNCIADO DEL PROBLEMA

\subsection{}

% poner aqui la SOLUCION DEL PROBLEMA	


\newpage{}


\end{document}