\documentclass{standalone}
\usepackage{mathpazo}
\usepackage{tikz}
\usepackage{circuitikz}
\usetikzlibrary{calc}

\begin{document}

\begin{circuitikz}[american resistors, american voltages, american currents]
\coordinate(A) at (5.5,3);
  \coordinate(B) at (3gmail.5,3);
  \draw
  (0,3) to [isource, l= $1A$, i=$I_1$] (0,0)
  (0,3) to [R, l=$2\Omega$, i=$I_2$] (3,3)
  (3,3) to [R, l=$1\Omega$, i=$I_3$] (3,0)
  (3,3) to [short, -*] (3.5,3)
  (5.5,3) to [short, *-, ] (6,3)
  (0,5) to [R,l=$2/3\Omega$, i=$I_4$] (6,5)
  (6,3) to [R, l= $4\Omega$, i=$I_5$] (6,0)
  (0,3) to [short, -] (0,5)
  (6,3) to [short, -] (6,5)
  (0,0) to [short,-] (6,0);
  \node[label=above:A] (A) at ($(A)$) {};
   \node[label=above:B] (B) at ($(B)$) {};
 %  \draw[thin, <-] (1.5,1.5)node{$I_a$}  ++(-60:0.5) arc (-60:170:0.5);
 %  \draw[thin, <-] (4.5,1.5)node{$I_2$}  ++(-60:0.5) arc (-60:170:0.5);
%   \draw[thin, <-] (7.5,2)node{$I_3$}  ++(-60:0.5) arc (-60:170:0.5);
\end{circuitikz}

\end{document}