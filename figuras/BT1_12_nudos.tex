\documentclass{standalone}
\usepackage{mathpazo}
\usepackage{tikz}
\usepackage{circuitikz}
\usetikzlibrary{calc}

\begin{document}

\begin{circuitikz}[american resistors, american voltages, american currents]
\coordinate (A) at (3,1);
\coordinate (B) at (8,1);
\coordinate (C) at (5.5,0);
  \draw
  (0,0) to [isource, l_= $2\,{A}$] (0,-3)
  (2,0) to [R, l=$9\Omega$, i=$I_1$] (2,-3)
  (0,0) to [short,-] (2,0)
  (0,-3) to [short,-] (2,-3)
  (1.,0) to [short,-] (1.,1)
  (1.,1) to [short,-] (3,1)
  (3,1) to [short,-, i=$I_6$](3,2)
  (3,1) to [short,-, i=$I_4$](3,0)
  (3,2) to [R, l=$20\Omega$] (8,2)
  (3,0) to [R, l=$20\Omega$] (5.5,0)
  (5.5,0) to [R, l=$20\Omega$, i^<=$I_5$] (8,0)
  (8,2) to [short,-](8,0)
  (5.5,0) to [R, l=$18\Omega$, i=$I_3$] (5.5,-5)
  (8,1) to [short,-] (10,1)
  (10,1) to [short,-] (10,-1)
  (9,-1) to [short,-] (11,-1)
  (9,-3) to [short,-] (11,-3)
  (10,-3) to [short,-] (10,-5)
  (9,-1) to [R,l_=$4\Omega$] (9,-3)
  (11,-3) to [isource, l_=$4\,A$] (11,-1)
  (10,-5) to [short,-,i=$I_2$] (5.5,-5)
  (5.5,-5) to [short,-,i^<=$I_7$] (1,-5)
  (1,-5) to [short,-] (1,-3)
  (5.5,-5) to [short, -] (5.5,-5) node[ground] {};
    \node[label=below left:A] (A) at ($(A)$) {};
    \node[label=below right:B] (B) at ($(B)$) {};
    \node[label=above:C] (C) at ($(C)$) {};
\end{circuitikz}

\end{document}