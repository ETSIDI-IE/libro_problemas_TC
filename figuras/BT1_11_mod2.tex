\documentclass{standalone}
\usepackage{mathpazo}
\usepackage{tikz}
\usepackage{circuitikz}
\usetikzlibrary{calc}

\begin{document}

\begin{circuitikz}[american resistors, american voltages]
\coordinate (D) at (3,0);
\coordinate (C) at (9,0);
\coordinate (B) at (6,5);
\coordinate (A) at (3,5);
  \draw
   (3,0) to [esource, v= $8\,V$] (3,5)
   (3,5) to [R, l=$4\Omega$, i=$I_3$] (6,5)
   (6,5) to [R, l=$2\Omega$] (6,2)
   (6,2) to [esource, v=$8\,V$, i=$I_4$] (6,0)
   (3,0) to [short,-, i^<=$I_7$] (6,0)
   (6,0) to [R, l=$1\Omega$, i^<=$I_6$] (9,0)
   (9,0) to [R, l=$2\Omega$] (9,5)
   (6,5) to [R, l=$1\Omega$, i=$I_5$] (9,5);

  \node[label=below:D] (D) at ($(D)$) {};
  \node[label=above:A] (A) at ($(A)$) {};
  \node[label=below:C] (C) at ($(C)$) {};
  \node[label=above:B] (B) at ($(B)$) {};
  \draw[thin, <-] (4.5,2.5)node{$I_b$}  ++(-60:0.5) arc (-60:170:0.5);
  \draw[thin, <-] (7.5,2.5)node{$I_c$}  ++(-60:0.5) arc (-60:170:0.5);

\end{circuitikz}

\end{document}
